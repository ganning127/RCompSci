%%% Research Diary - Entry
%%% Template by Mikhail Klassen, April 2013

\documentclass[11pt,letterpaper]{article}

\newcommand{\workingDate}{\textsc{2022 $|$ January $|$ 31}}
\newcommand{\userName}{Ganning Xu}
\newcommand{\institution}{Research in \\ Computational Science \\ North Carolina School of Science and Math}
\usepackage{researchdiary_png}
%\usepackage{natbib}
%\bibliographystyle{abbrvnat}
%\setcitestyle{authoryear,open={((},close={))}}
\usepackage{float}
\usepackage{wrapfig}
\usepackage{hyperref}
\usepackage{hyperref}

\begin{comment}
Resources
    Table Generator: https://www.tablesgenerator.com/
    Overleaf Project: https://www.overleaf.com/project/61fc1906472d003c89b4c1f1
    GitHub Project: https://github.com/ganning127/RCompSci_Research_Notebook
\end{comment}

\begin{document}
\univlogo

\section{Week of January 31, 2022}

\subsection{Monday, January 31, 2022}
First day of research in computational science. Went over introductions, course expectations, and how this class would work.

\subsection{Wednesday, February 2, 2022}
Went over how to read request for proposals (RFP) and covered the Research in Computational Science \href{https://drive.google.com/file/d/1cvPQnz40H3bqiEyOdgzRMyYjtK4gxPB5/view?usp=sharing}{RFP}.


We also covered an intro \href{https://drive.google.com/file/d/1hlGEsI95i6WEo5NzAbqjr73ax-MP5L8G/view?usp=sharing}{guide to computational thinking}. Begin by starting with a research question that can be broken down into subproblems. The first subproblem should be "low hanging fruit", while the latter ones should be harder to achieve. Also, it is important to make assumptions when creating the model, as it simplifies it. It's important to be able to distinguish which data is important and which data is not as useful. 

When creating algorithms, you can either use an existing one, modify an existing one, or create your own. Creating your own algorithm is usually difficult. 

\textbf{Important notes}
\begin{itemize}
    \item Letter of intent due on Friday, March 4, 2022 at 5 PM. 
    \item Preliminary proposal due on Friday, March 25, 2022, at 5 PM. 
    \item Full proposal due Wednesday, April 20, at 5 PM. 
\end{itemize}

\subsection{Thursday, February 3, 2022}
Learned \LaTeX \\

Table \ref{tab:ideas} shows a table of some of my ideas for my research work.

\begin{table}[H]
\begin{center}
\begin{tabular}{|c|c|c|c|}
\hline
\textbf{Topic idea} & \textbf{\begin{tabular}[c]{@{}c@{}}Technology\\ Computational "X"\end{tabular}} & \textbf{Techniques and Tools} \\ \hline
Heart disease detection algorithm & Machine Learning & TensorFlow, Keras, Kaggle \\ \hline
Long text simplification & NLP & GTP3 and Kaggle \\ \hline
Factors of a successful election & Machine Learning & TensorFlow, Keras, Large dataset \\ \hline
\end{tabular}
\caption{Table of CT Ideas}
\label{tab:ideas}
\end{center}
\end{table}

\subsection{Friday, February 4, 2022}

\subsubsection * {How Science Works}
Today we learned about how science works from the national level to a researcher when they submit their RFP. Here is a link to my \href{https://docs.google.com/document/d/17-9rgkjDyZuy95PmMjWT7NfPlYd96HXTcVR5unhhW_A/edit?usp=sharing}{notes}. 

Basically, the flow of power works like this: NSB \rightarrow NSF \rightarrow Directorates \rightarrow Groups \rightarrow RFP. 

Once a researcher sends in an RFP:
\begin{enumerate}
    \item The RFP is sent to a specific project officer (typically university faculty on loan)
    \item The Project Officer will then create a certain number of review teams (~5 in team team), to review ~20 proposals.
    \item Each person in a review team scores each proposal (excellent, very good, good, fair, poor)
    \item After individual review, each review team goes to Washington DC and works on reviewing proposals for ~3 days in a review panel together
    \item Each proposal is ranked into (High, medium, and low) priority for funding. This recommendation goes to the NSF project officer, who makes a decision on who gets the money. They can overrule a panel.
\end{enumerate}

When an RFP is approved, the person in charge of the research is the Principal Investigator (PI). Each PI usually has CoPI's, unless the project is really small. 

\subsubsection * {Example RFP}
Today, we also looked over an example RFP that Mr. Gotwals submitted. It is important to note that the project summary can only be one page long and the project description has to be less than 15 pages. 

\subsection{Saturday, February 5, 2022}
Today I read over the example proposal that Mr. Gotwals gave us and completed the reading check that covered the podcast and the Intro to Computational Thinking chapter. I also started on the content from the proposal template today.

\subsection{Sunday, February 6, 2022}
Read over \href{https://drive.google.com/file/d/1weJUCV1x84bmy6dvqFjRlOviiD8dVDjg/view?usp=sharing}{GotwalsGuideCT.pdf}. Also read over \href{https://drive.google.com/file/d/1p4ew41jI2WH81WCaB8ADkFOrTVsac7Wl/view?usp=sharing}{How to summarize a research paper}.

\newpage
\begin{thebibliography}{9}
\end{thebibliography}


\end{document}

\begin{comment}
Sample LaTeX Code
    % \bibitem{giusti}
    % Giusti, Santochi, \emph{Tecnologia Meccanica e Studi di Fabbricazione}. Casa Editrice Ambrosiana, Seconda Edizione
    
    
    % \begin{figure}[H]\centering
    % \includegraphics[scale=0.5]{weintrop.jpg}
    % \caption{Weintrop's Taxonomy}
    % \label{fig:weintrop}
    % \end{figure}

\end{comment}
