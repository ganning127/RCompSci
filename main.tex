%%% Research Diary - Entry
%%% Template by Mikhail Klassen, April 2013
%%% TODO (update things from first day of class, think of ideas to fill in table below)

\documentclass[11pt,letterpaper]{article}

\newcommand{\workingDate}{\textsc{2022 $|$ February $|$ 3}}
\newcommand{\userName}{Ganning Xu}
\newcommand{\institution}{Research in \\ Computational Science \\ North Carolina School of Science and Math}
\usepackage{researchdiary_png}
%\usepackage{natbib}
%\bibliographystyle{abbrvnat}
%\setcitestyle{authoryear,open={((},close={))}}
\usepackage{float}
\usepackage{wrapfig}
\usepackage{hyperref}

\begin{comment}
Resources
    Table Generator: https://www.tablesgenerator.com/
    Overleaf Project: https://www.overleaf.com/project/61fc1906472d003c89b4c1f1
\end{comment}


\begin{document}
\univlogo

\section{Week of January 31, 2022}

\subsection{Monday, January 31, 2022}
First day of research in computational science. Went over introductions, course expectations, and how this class would work.

\subsection{Wednesday, February 2, 2022}
Went over how to read request for proposals (RFP) and the specific program requirements for this course. 

\textbf{Important notes}
\begin{itemize}
    \item Letter of intent due on Friday, March 4, 2022 at 5 PM. 
    \item Preliminary proposal due on Friday, March 25, 2022, at 5 PM. 
    \item Full proposal due Wednesday, April 20, at 5 PM. 
\end{itemize}

\subsection{Thursday, February 3, 2022}
Started coding Research Diary in \LaTeX. 

Table \ref{tab:ideas} shows a table of some of my ideas for my research work.

\begin{table}[H]
\begin{center}
\begin{tabular}{|c|c|c|c|}
\hline
\textbf{Topic idea} & \textbf{\begin{tabular}[c]{@{}c@{}}Technology\\ Computational "X"\end{tabular}} & \textbf{Techniques} & \textbf{Tools} \\ \hline
Idea 1 & Technology 1 & Technique 1 & Tool 1 \\ \hline
Idea 2 & Technology 2 & Technique 2 & Tool 2 \\ \hline
Idea 3 & Technology 3 & Technique 3 & Tool 3 \\ \hline
\end{tabular}
\caption{Table of CT Ideas}
\label{tab:ideas}
\end{center}
\end{table}





\newpage
\begin{thebibliography}{9}



\end{thebibliography}


\end{document}

\begin{comment}
Sample LaTeX Code
    % \bibitem{giusti}
    % Giusti, Santochi, \emph{Tecnologia Meccanica e Studi di Fabbricazione}. Casa Editrice Ambrosiana, Seconda Edizione
    
    
    % \begin{figure}[H]\centering
    % \includegraphics[scale=0.5]{weintrop.jpg}
    % \caption{Weintrop's Taxonomy}
    % \label{fig:weintrop}
    % \end{figure}

\end{comment}
