%%% Research Diary - Entry
%%% Template by Mikhail Klassen, April 2013

\documentclass[11pt,letterpaper]{article}

\newcommand{\workingDate}{\textsc{2022 $|$ January $|$ 31}}
\newcommand{\userName}{Ganning Xu}
\newcommand{\institution}{Research in \\ Computational Science \\ North Carolina School of Science and Math}
\usepackage{researchdiary_png}
%\usepackage{natbib}
%\bibliographystyle{abbrvnat}
%\setcitestyle{authoryear,open={((},close={))}}
\usepackage{float}
\usepackage{wrapfig}
\usepackage{hyperref}
\usepackage{hyperref}

\begin{comment}
Resources
    Table Generator: https://www.tablesgenerator.com/
    Overleaf Project: https://www.overleaf.com/project/61fc1906472d003c89b4c1f1
    GitHub Project: https://github.com/ganning127/RCompSci_Research_Notebook
\end{comment}

\begin{document}
\univlogo

\section{Week of January 31, 2022}

$\surd$ {\Large \textcolor{red}{RRG Comment:acknowledging notebook establishment } } 

\subsection{Monday, January 31, 2022}
First day of research in computational science. Went over introductions, course expectations, and how this class would work.

\subsection{Wednesday, February 2, 2022}
Went over how to read request for proposals (RFP) and covered the Research in Computational Science \href{https://drive.google.com/file/d/1cvPQnz40H3bqiEyOdgzRMyYjtK4gxPB5/view?usp=sharing}{RFP}.


We also covered an intro \href{https://drive.google.com/file/d/1hlGEsI95i6WEo5NzAbqjr73ax-MP5L8G/view?usp=sharing}{guide to computational thinking}. Begin by starting with a research question that can be broken down into subproblems. The first subproblem should be "low hanging fruit", while the latter ones should be harder to achieve. Also, it is important to make assumptions when creating the model, as it simplifies it. It's important to be able to distinguish which data is important and which data is not as useful. 

When creating algorithms, you can either use an existing one, modify an existing one, or create your own. Creating your own algorithm is usually difficult. 

\textbf{Important notes}
\begin{itemize}
    \item Letter of intent due on Friday, March 4, 2022 at 5 PM. 
    \item Preliminary proposal due on Friday, March 25, 2022, at 5 PM. 
    \item Full proposal due Wednesday, April 20, at 5 PM. 
\end{itemize}

\subsection{Thursday, February 3, 2022}
Learned \LaTeX 

Table \ref{tab:ideas} shows a table of some of my ideas for my research work.

\begin{table}[H]
\begin{center}
\begin{tabular}{|c|c|c|c|}
\hline
\textbf{Topic idea} & \textbf{\begin{tabular}[c]{@{}c@{}}Technology\\ Computational "X"\end{tabular}} & \textbf{Techniques and Tools} \\ \hline
Heart disease detection algorithm & Machine Learning & TensorFlow, Keras, Kaggle \\ \hline
Long text simplification & NLP & GTP3 and Kaggle \\ \hline
Factors of a successful election & Machine Learning & TensorFlow, Keras, Large dataset \\ \hline
\end{tabular}
\caption{Table of CT Ideas}
\label{tab:ideas}
\end{center}
\end{table}

\subsection{Friday, February 4, 2022}

\subsubsection * {How Science Works}
Today we learned about how science works from the national level to a researcher when they submit their RFP. Here is a link to my \href{https://docs.google.com/document/d/17-9rgkjDyZuy95PmMjWT7NfPlYd96HXTcVR5unhhW_A/edit?usp=sharing}{notes}. 

Basically, the flow of power works like this: NSB $\rightarrow$ NSF $\rightarrow$ Directorates $\rightarrow$ Groups $\rightarrow$ RFP. 

Once a researcher sends in an RFP:
\begin{enumerate}
    \item The RFP is sent to a specific project officer (typically university faculty on loan)
    \item The Project Officer will then create a certain number of review teams (~5 in team team), to review ~20 proposals.
    \item Each person in a review team scores each proposal (excellent, very good, good, fair, poor)
    \item After individual review, each review team goes to Washington DC and works on reviewing proposals for ~3 days in a review panel together
    \item Each proposal is ranked into (High, medium, and low) priority for funding. This recommendation goes to the NSF project officer, who makes a decision on who gets the money. They can overrule a panel.
\end{enumerate}

When an RFP is approved, the person in charge of the research is the Principal Investigator (PI). Each PI usually has CoPI's, unless the project is really small. 

\subsubsection * {Example RFP}
Today, we also looked over an example RFP that Mr. Gotwals submitted. It is important to note that the project summary can only be one page long and the project description has to be less than 15 pages. 

\subsection{Saturday, February 5, 2022}
Today I read over the example proposal that Mr. Gotwals gave us and completed the reading check that covered the podcast and the Intro to Computational Thinking chapter. I also started on the content from the proposal template today.

\subsection{Sunday, February 6, 2022}
Read over \href{https://drive.google.com/file/d/1weJUCV1x84bmy6dvqFjRlOviiD8dVDjg/view?usp=sharing}{GotwalsGuideCT.pdf}. Also read over \href{https://drive.google.com/file/d/1p4ew41jI2WH81WCaB8ADkFOrTVsac7Wl/view?usp=sharing}{How to summarize a research paper}.

\section{Week of February 7, 2022}
\subsection{Monday, February 7, 2022}
Learned about how spreadsheets worked in class. Completed three spreadsheet labs from the Bohr model, Gaussian Distribution, and Diatomic molecules. 

\subsection{Tuesday, February 8, 2022}
Corrected one of the spreadsheet labs.

\subsection{Wednesday, February 9, 2022}
\href{https://www.ncssm.edu/library}{ncssm.edu/library} is a good source for research articles. The \href{https://ncssm.follettdestiny.com/common/servlet/presenthomeform.do?l2m=Home&tm=Home&l2m=Home}{full catalog} has acess to journals.

\begin{itemize}
    \item HeinOnline - Legal
    \item Liebert Medical Journals - Medical Research
    \item JSTOR - Humanities
    \item NC Live - Biggest DB (UNC System, private colleges, community colleges). Can access the NYTimes and Washington post through here
\end{itemize}

\textbf{Ways to read a research paper}
\begin{enumerate}
    \item Read abstract and conclusions. If important can look at the entire thing. 
    \item The last sentence of the introduction usually contains the RQ
    \item Look at references of research paper first (see if you see a certain name that is repeated, they are the expert)
    \item Read many papers from the expert and see if you can reach out to them for them to be your mentor
    \item Keep the \href{https://www.mendeley.com/reference-manager/}{citation manager} UPDATED!
\end{enumerate}

\textbf{Types of Research}
\begin{itemize}
    \item Literature Review: A researcher goes out and finds research papers about a topic over a certain period of time. This is very good to find if you can. 
\end{itemize}


\subsection{Thursday, February 10, 2022}
Today I read over the research paper: Recent Approaches for Text Summarization Notes, and created a draft of my J-Club presentation on the topic. There are two main types of text summarization: abstractive and extractive. Abstractive summarization focuses on semantic understanding the text and re-expressing that understanding in easier words. Extractive summarization focuses on removing unnecessary sentences and words to create a summary sentence with parts of existing ones.

\subsection{Friday, February 11, 2022}
Today was the Mathematica lab, where we learned how to perform text analysis on a speech that was 80,000 characters long. I also made edits to my J-Club draft today.

\subsection{Saturday, February 12, 2022}
I was at the NCHSAA States Swim meet for today, so I didn't get a chance to work on RCompSci.

\newpage


\end{document}

\begin{comment}
%Sample LaTeX Code
    % \bibitem{giusti}
    % Giusti, Santochi, \emph{Tecnologia Meccanica e Studi di Fabbricazione}. Casa Editrice Ambrosiana, Seconda Edizione
    
    
    % \begin{figure}[H]\centering
    % \includegraphics[scale=0.5]{weintrop.jpg}
    % \caption{Weintrop's Taxonomy}
    % \label{fig:weintrop}
    % \end{figure}

\end{comment}
